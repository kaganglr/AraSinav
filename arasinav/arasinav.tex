% Options for packages loaded elsewhere
\PassOptionsToPackage{unicode}{hyperref}
\PassOptionsToPackage{hyphens}{url}
\PassOptionsToPackage{dvipsnames,svgnames,x11names}{xcolor}
%
\documentclass[
  12pt,
]{article}
\usepackage{amsmath,amssymb}
\usepackage{lmodern}
\usepackage{iftex}
\ifPDFTeX
  \usepackage[T1]{fontenc}
  \usepackage[utf8]{inputenc}
  \usepackage{textcomp} % provide euro and other symbols
\else % if luatex or xetex
  \usepackage{unicode-math}
  \defaultfontfeatures{Scale=MatchLowercase}
  \defaultfontfeatures[\rmfamily]{Ligatures=TeX,Scale=1}
\fi
% Use upquote if available, for straight quotes in verbatim environments
\IfFileExists{upquote.sty}{\usepackage{upquote}}{}
\IfFileExists{microtype.sty}{% use microtype if available
  \usepackage[]{microtype}
  \UseMicrotypeSet[protrusion]{basicmath} % disable protrusion for tt fonts
}{}
\makeatletter
\@ifundefined{KOMAClassName}{% if non-KOMA class
  \IfFileExists{parskip.sty}{%
    \usepackage{parskip}
  }{% else
    \setlength{\parindent}{0pt}
    \setlength{\parskip}{6pt plus 2pt minus 1pt}}
}{% if KOMA class
  \KOMAoptions{parskip=half}}
\makeatother
\usepackage{xcolor}
\usepackage[margin=1in]{geometry}
\usepackage{longtable,booktabs,array}
\usepackage{calc} % for calculating minipage widths
% Correct order of tables after \paragraph or \subparagraph
\usepackage{etoolbox}
\makeatletter
\patchcmd\longtable{\par}{\if@noskipsec\mbox{}\fi\par}{}{}
\makeatother
% Allow footnotes in longtable head/foot
\IfFileExists{footnotehyper.sty}{\usepackage{footnotehyper}}{\usepackage{footnote}}
\makesavenoteenv{longtable}
\usepackage{graphicx}
\makeatletter
\def\maxwidth{\ifdim\Gin@nat@width>\linewidth\linewidth\else\Gin@nat@width\fi}
\def\maxheight{\ifdim\Gin@nat@height>\textheight\textheight\else\Gin@nat@height\fi}
\makeatother
% Scale images if necessary, so that they will not overflow the page
% margins by default, and it is still possible to overwrite the defaults
% using explicit options in \includegraphics[width, height, ...]{}
\setkeys{Gin}{width=\maxwidth,height=\maxheight,keepaspectratio}
% Set default figure placement to htbp
\makeatletter
\def\fps@figure{htbp}
\makeatother
\setlength{\emergencystretch}{3em} % prevent overfull lines
\providecommand{\tightlist}{%
  \setlength{\itemsep}{0pt}\setlength{\parskip}{0pt}}
\setcounter{secnumdepth}{5}
\newlength{\cslhangindent}
\setlength{\cslhangindent}{1.5em}
\newlength{\csllabelwidth}
\setlength{\csllabelwidth}{3em}
\newlength{\cslentryspacingunit} % times entry-spacing
\setlength{\cslentryspacingunit}{\parskip}
\newenvironment{CSLReferences}[2] % #1 hanging-ident, #2 entry spacing
 {% don't indent paragraphs
  \setlength{\parindent}{0pt}
  % turn on hanging indent if param 1 is 1
  \ifodd #1
  \let\oldpar\par
  \def\par{\hangindent=\cslhangindent\oldpar}
  \fi
  % set entry spacing
  \setlength{\parskip}{#2\cslentryspacingunit}
 }%
 {}
\usepackage{calc}
\newcommand{\CSLBlock}[1]{#1\hfill\break}
\newcommand{\CSLLeftMargin}[1]{\parbox[t]{\csllabelwidth}{#1}}
\newcommand{\CSLRightInline}[1]{\parbox[t]{\linewidth - \csllabelwidth}{#1}\break}
\newcommand{\CSLIndent}[1]{\hspace{\cslhangindent}#1}
\usepackage{polyglossia}
\setmainlanguage{turkish}
\usepackage{booktabs}
\usepackage{caption}
\captionsetup[table]{skip=10pt}
\ifLuaTeX
  \usepackage{selnolig}  % disable illegal ligatures
\fi
\IfFileExists{bookmark.sty}{\usepackage{bookmark}}{\usepackage{hyperref}}
\IfFileExists{xurl.sty}{\usepackage{xurl}}{} % add URL line breaks if available
\urlstyle{same} % disable monospaced font for URLs
\hypersetup{
  pdftitle={Yakınsama Hipotezine Eleştirel Bakış},
  pdfauthor={Kağan Güler},
  colorlinks=true,
  linkcolor={Maroon},
  filecolor={Maroon},
  citecolor={Blue},
  urlcolor={blue},
  pdfcreator={LaTeX via pandoc}}

\title{Yakınsama Hipotezine Eleştirel Bakış}
\author{Kağan Güler\footnote{21080360, \href{https://github.com/kaganglr/AraSinav}{Github Repo}}}
\date{}

\begin{document}
\maketitle

\hypertarget{giriux15f}{%
\section{Giriş}\label{giriux15f}}

\begin{verbatim}
                    Solow Modeli ve Yakınsama Hipotezi
\end{verbatim}

Ana akım iktisat\texttt{ın\ büyüme\ modeli\ olan\ Solow-Swan\ modelinin\ ülkeler\ arasındaki\ gelir\ eşitsizliklerini\ açıklama\ gücü\ tartışılır,\ ama\ (Mankiw-Romer-Weil,\ 1992)}in ana akım büyüme modeline uyarladıkları genişletilmiş modeli, eşitsizlikleri açıkladığını iddia etmiş ve uyarlanan bu model geniş bir kabul görmüştür.
Ülkeler arasındaki büyük gelir farklılıklarının zaman içinde birbirlerine yaklaşmadığı müşahade edildiğinden teorinin ampirik olarak yanlışlandığı düşünülmüş ve Solow-Swan modelinden çıkış eğilimleri kuvvetlenmişti.
Makalenin yaptığı ana katkı orijinal Solow-Swan modeline insan sermayesini dâhil ederek temel modelle gelir farklılıklarını koşullu yakınsamayla açıklayabilmesidir.
Bu istikamette (Acemoğlu, 2009: 109-112) büyüme\texttt{nin\ sebeplerini\ yakın\ sebepler\ (Proximate\ Causes)\ ve\ temel\ sebepler\ (Fundamental\ Causes)\ olarak\ ikiye\ ayırmış,\ yakın\ sebepleri\ genişletilmiş\ Solow-Swan\ modelindeki\ açıklayıcı\ ve\ dışsal\ değişkenler\ (fiziksel\ sermaye-tasarruf,\ beşeri\ sermaye,\ teknoloji\ ve\ nüfus)\ olarak\ tasnif\ etmiş,\ temel\ sebepler\ olarak\ da\ şans,\ coğrafya,\ kültür\ ve\ kurumları\ göstermiştir.\ Dolayısıyla,\ MRW}nin katkılarıyla, ilgi genişletilmiş Solow-Swan modelinin açıklayıcı değişkenleri yerine bu açıklayıcı değişkenleri belirleyen temel sebeplere kaymıştır.
Ders kitabı Solow-Swan modelinin MRW tarafından genişletilerek karmaşık olan gerçek hayatı daha iyi tasvir eder hale getirilmesi, modelin içsel büyüme modelleri lehine tasfiye edilmesini önlemiş ve ana akım araştırmalarının temel modeli olarak kalmasını sağlamıştır.
Bu çalışmada MRW izlenerek Solow-Swan Modeli (Textbook Slow-Swan Model), Genişletilmiş Solow-Swan Modeli (Augmented Solow-Swan Model), Mutlak ve Koşullu Yakınsama hipotezleri (Absolute and Conditional Convergence) panel veri sabit etkiler (Fixed Effects) yöntemiyle test edilmektedir.
Dolayısıyla, bu çalışmamız, ele alınan dönemin yakınlığı, kullanılan değişkenlerin tamamen ana akım modelini esas alması, panel veri kullanılması, gözlem sayısının çokluğu ve özellikle oluşturulan yeni ülke gruplarıyla uygulamalı literatüre önemli bir katkı sağlamaktadır.
Üçüncü bölümde ekonometrik model değişkenlerinin ana akım modelini yansıtabilmesi için MRW\texttt{nin\ kısa\ bir\ özeti\ verilmektedir.\ Sonraki\ üç\ bölümde\ Solow-Swan\ Modeli\ ve\ Genişletilmiş\ Solow-Swan\ Modelleri\ kurulmakta\ ve\ ampirik\ modelle\ veri\ seti\ açıklanmaktadır.\ Solow-Swan\ büyüme\ modeli\ olarak\ anılan\ bu\ orijinal\ neoklasik\ model,\ ülkelerin\ işçi\ başına\ düşen\ gelirlerinin\ zaman\ içinde\ aynı\ düzeye\ geleceğini\ öngörmekteydi.\ Yakınsama\ başarısızlığının\ etkisiyle\ (Barro,\ 1989)\ Solow-Swan\ modelindeki\ yapıyı\ terk\ ederek\ büyüme,\ yatırım\ ve\ doğurganlık\ gibi\ pek\ çok\ değişken\ arasındaki\ eşanlı\ ilişkilere\ Sarıbaş,\ H.\ Regresyona\ uygun\ olan\ ve\ doğrudan\ Solow-Swan\ modelinden\ üretilen\ ilk\ model\ MRW}nin çalışmasıdır ve bu çalışma ana akım iktisadının o zamandan beri en temel büyüme regresyonu olarak kabul edilmiştir (Islam, 2003: 319).
Genişletilmiş Solow-Swan modeli olarak anılan bu modele göre yakınsama koşulludur.
MRW\texttt{nin\ çalışmasından\ sonra\ Solow-Swan\ modelinin\ ekonometrik\ olarak\ daha\ iyi\ nasıl\ tahmin\ edilebileceği\ üzerinde\ durulmuş,\ esas\ model\ olan\ Solow-Swan}ın teorik üstünlüğü ana akım tarafından devam ettirilmiştir.Bu konuda detaylı bilgi almak için atıf yaptığım Saribaş (\protect\hyperlink{ref-saribacs2016ana}{2016}) okunabilir.

\begin{verbatim}
    KIRILGAN BEŞLİ EKONOMİLERİ İÇİN YAKINSAMA HİPOTEZİNİN GEÇERLİLİĞİ  Öz Yakınsama hipotezi
    
\end{verbatim}

Neoklasik teoriye dayanan Solow büyüme modeli temelli, göreli yoksul ülkelerin zengin ülkelere kıyasla daha hızlı büyüyeceği varsayımına dayanmaktadır.
Çalışmada kırılgan beşli ülkelerinin kişi başı GSYH\texttt{lerinin\ birbirlerine\ ve\ grup\ içindeki\ lider\ ülkeye\ önce\ topluca\ ardından\ bireysel\ olarak\ yakınsayıp\ yakınsamadığı,\ 1980-2017\ örneklem\ dönemi\ için\ çağdaş\ panel\ birim\ kök\ testleriyle\ araştırılmıştır.\ Lider\ ülke\ olarak\ seçilen\ Güney\ Afrika}ya ülkelerin, Znlamlılık seviyesinde kuvvetli yakınsama gösterdikleri tespit edilmiştir.

Göreceli gelirlerdeki büyük değişikliklerin en çarpıcı örnekleri büyüme mucizeleridir.
Ekonomik büyüme yazınının en çok tartışılan ve araştırma alanı bulan konularından olan yakınsama hipotezi aslında Solow modelinin genel çıkarımıdır.
Az gelişmiş veya gelişmekte olan ülkelerin verimlilikteki artışları, gelişmiş olan ülkelere nazaran daha hızlı olacağı ve bu yolla büyüme farklılıklarının zaman içinde azalıp yakınsamanın gerçekleşeceği ileri sürülmektedir.
Bu noktada, kişi başı geliri düşük olduğu ülkelerin, gelişmiş ülkelere kıyasla daha hızlı büyüyerek onları yakalaması beklenmektedir. Daha detaylı bilgi için : YILMAZ ve KESBİÇ (\protect\hyperlink{ref-yilmaz2020kirilgan}{2020}) makalesi okunabilir.

\hypertarget{uxe7alux131ux15fmanux131n-amacux131}{%
\subsection{Çalışmanın Amacı}\label{uxe7alux131ux15fmanux131n-amacux131}}

Yakınsama Hipotezi ile ilgili topladığım makalelerle bu hipotezin doğruluğu hakkında eleştirel bir bakış ortaya sunmaktır.

\hypertarget{literatuxfcr}{%
\subsection{Literatür}\label{literatuxfcr}}

Makaleler sırasıyla ; (\protect\hyperlink{ref-ceylan2010yakinsama}{\textbf{ceylan2010yakinsama?}}), Saribaş (\protect\hyperlink{ref-saribacs2016ana}{2016}), Lichtenberg (\protect\hyperlink{ref-lichtenberg1994testing}{1994}), Rassekh (\protect\hyperlink{ref-rassekh1998convergence}{1998}).

\newpage

\hypertarget{references}{%
\section{Kaynakça}\label{references}}

\hypertarget{refs}{}
\begin{CSLReferences}{1}{0}
\leavevmode\vadjust pre{\hypertarget{ref-lichtenberg1994testing}{}}%
Lichtenberg, F. R. (1994). Testing the convergence hypothesis. \emph{The Review of Economics and Statistics}, 576-579.

\leavevmode\vadjust pre{\hypertarget{ref-rassekh1998convergence}{}}%
Rassekh, F. (1998). The convergence hypothesis: History, theory, and evidence. \emph{Open economies review}, \emph{9}, 85-105.

\leavevmode\vadjust pre{\hypertarget{ref-saribacs2016ana}{}}%
Saribaş, H. (2016). Ana ak{ı}m b{ü}y{ü}me modeli ve yak{ı}nsama hipotezlerinin analizi: panel veri yakla{ş}{ı}m{ı}. \emph{Sosyoekonomi}, \emph{24}(30), 169-186.

\leavevmode\vadjust pre{\hypertarget{ref-yilmaz2020kirilgan}{}}%
YILMAZ, M. ve KESBİÇ, C. Y. (2020). KIRILGAN BE{Ş}L{İ} EKONOM{İ}LER{İ} {İ}{Ç}{İ}N YAKINSAMA H{İ}POTEZ{İ}N{İ}N GE{Ç}ERL{İ}L{İ}{Ğ}{İ}. \emph{Elektronik Sosyal Bilimler Dergisi}, \emph{19}(75), 1275-1293.

\end{CSLReferences}

\end{document}
