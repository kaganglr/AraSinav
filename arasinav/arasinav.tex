% Options for packages loaded elsewhere
\PassOptionsToPackage{unicode}{hyperref}
\PassOptionsToPackage{hyphens}{url}
\PassOptionsToPackage{dvipsnames,svgnames,x11names}{xcolor}
%
\documentclass[
  12pt,
]{article}
\usepackage{amsmath,amssymb}
\usepackage{lmodern}
\usepackage{iftex}
\ifPDFTeX
  \usepackage[T1]{fontenc}
  \usepackage[utf8]{inputenc}
  \usepackage{textcomp} % provide euro and other symbols
\else % if luatex or xetex
  \usepackage{unicode-math}
  \defaultfontfeatures{Scale=MatchLowercase}
  \defaultfontfeatures[\rmfamily]{Ligatures=TeX,Scale=1}
\fi
% Use upquote if available, for straight quotes in verbatim environments
\IfFileExists{upquote.sty}{\usepackage{upquote}}{}
\IfFileExists{microtype.sty}{% use microtype if available
  \usepackage[]{microtype}
  \UseMicrotypeSet[protrusion]{basicmath} % disable protrusion for tt fonts
}{}
\makeatletter
\@ifundefined{KOMAClassName}{% if non-KOMA class
  \IfFileExists{parskip.sty}{%
    \usepackage{parskip}
  }{% else
    \setlength{\parindent}{0pt}
    \setlength{\parskip}{6pt plus 2pt minus 1pt}}
}{% if KOMA class
  \KOMAoptions{parskip=half}}
\makeatother
\usepackage{xcolor}
\usepackage[margin=1in]{geometry}
\usepackage{longtable,booktabs,array}
\usepackage{calc} % for calculating minipage widths
% Correct order of tables after \paragraph or \subparagraph
\usepackage{etoolbox}
\makeatletter
\patchcmd\longtable{\par}{\if@noskipsec\mbox{}\fi\par}{}{}
\makeatother
% Allow footnotes in longtable head/foot
\IfFileExists{footnotehyper.sty}{\usepackage{footnotehyper}}{\usepackage{footnote}}
\makesavenoteenv{longtable}
\usepackage{graphicx}
\makeatletter
\def\maxwidth{\ifdim\Gin@nat@width>\linewidth\linewidth\else\Gin@nat@width\fi}
\def\maxheight{\ifdim\Gin@nat@height>\textheight\textheight\else\Gin@nat@height\fi}
\makeatother
% Scale images if necessary, so that they will not overflow the page
% margins by default, and it is still possible to overwrite the defaults
% using explicit options in \includegraphics[width, height, ...]{}
\setkeys{Gin}{width=\maxwidth,height=\maxheight,keepaspectratio}
% Set default figure placement to htbp
\makeatletter
\def\fps@figure{htbp}
\makeatother
\setlength{\emergencystretch}{3em} % prevent overfull lines
\providecommand{\tightlist}{%
  \setlength{\itemsep}{0pt}\setlength{\parskip}{0pt}}
\setcounter{secnumdepth}{5}
\newlength{\cslhangindent}
\setlength{\cslhangindent}{1.5em}
\newlength{\csllabelwidth}
\setlength{\csllabelwidth}{3em}
\newlength{\cslentryspacingunit} % times entry-spacing
\setlength{\cslentryspacingunit}{\parskip}
\newenvironment{CSLReferences}[2] % #1 hanging-ident, #2 entry spacing
 {% don't indent paragraphs
  \setlength{\parindent}{0pt}
  % turn on hanging indent if param 1 is 1
  \ifodd #1
  \let\oldpar\par
  \def\par{\hangindent=\cslhangindent\oldpar}
  \fi
  % set entry spacing
  \setlength{\parskip}{#2\cslentryspacingunit}
 }%
 {}
\usepackage{calc}
\newcommand{\CSLBlock}[1]{#1\hfill\break}
\newcommand{\CSLLeftMargin}[1]{\parbox[t]{\csllabelwidth}{#1}}
\newcommand{\CSLRightInline}[1]{\parbox[t]{\linewidth - \csllabelwidth}{#1}\break}
\newcommand{\CSLIndent}[1]{\hspace{\cslhangindent}#1}
\usepackage{polyglossia}
\setmainlanguage{turkish}
\usepackage{booktabs}
\usepackage{caption}
\captionsetup[table]{skip=10pt}
\ifLuaTeX
  \usepackage{selnolig}  % disable illegal ligatures
\fi
\IfFileExists{bookmark.sty}{\usepackage{bookmark}}{\usepackage{hyperref}}
\IfFileExists{xurl.sty}{\usepackage{xurl}}{} % add URL line breaks if available
\urlstyle{same} % disable monospaced font for URLs
\hypersetup{
  pdftitle={Yakınsama Hipotezi},
  pdfauthor={Kağan Güler},
  colorlinks=true,
  linkcolor={Maroon},
  filecolor={Maroon},
  citecolor={Blue},
  urlcolor={blue},
  pdfcreator={LaTeX via pandoc}}

\title{Yakınsama Hipotezi}
\author{Kağan Güler\footnote{21080360, \href{https://github.com/kaganglr/AraSinav}{Github Repo}}}
\date{}

\begin{document}
\maketitle

\hypertarget{giriux15f}{%
\section{Giriş}\label{giriux15f}}

Bu taslak size proje ödevinizi yazarken yardımcı olması amacıyla oluşturulmuştur. Ödevlerinizde, makalelerinizde, sunumlarınızda ve projelerinizde kullandığınız tüm bilgi kaynaklarına atıfta bulunmalısınız. Alıntı ve gönderme yapmak okuyuculara çalışmanızda kullandığınız/başvurduğunuz kaynaklara ulaşma imkanı sağlar. \textbf{Her ne kadar kendi sözlerinizi kullanıyor olsanız da, başkalarına ait fikirleri çalışmanızda aktarıyorsanız bu fikirlerin kaynağını belgelemek zorundasınız. Aksi takdirde akademik intihal yapmış olursunuz.} Yazım konusunda Aydınonat (\protect\hyperlink{ref-aydinonat:2007}{2007})'ye başvurabilirsiniz.

Bu yeni makale atıfım (\protect\hyperlink{ref-ceylan2010yakinsama}{Ceylan, 2010})

Proje ödevinizde yer alan başlıkların bu metinde yer alan başlıkları kesinlikle içermesi gerekmektedir (doğal olarak ilk bölüm başlığı hariç). Burada kullanılan başlıklar haricinde farklı alt başlıklar da kullanabilirsiniz. Projenizi yazarken bu dosyayı taslak olarak kullanıp içeriğini projenize uyarlayınız.

\hypertarget{uxe7alux131ux15fmanux131n-amacux131}{%
\subsection{Çalışmanın Amacı}\label{uxe7alux131ux15fmanux131n-amacux131}}

Bu bölümde yaptığınız çalışmanın amacından ve öneminden birkaç paragraf ile bahsediniz.

\hypertarget{literatuxfcr}{%
\subsection{Literatür}\label{literatuxfcr}}

Bu bölümde konu ile ilgili olarak okuduğunuz makaleleri referans vererek tartışınız. \textbf{Her makaleyi ayrı başlık altında tek tek özetlemeyiniz.} Literatür taramasında \textbf{en az dört} makaleye atıf yapılması ve bu makalelerden \textbf{en az ikisinin İngilizce} olması gerekmektedir.

\textbf{\emph{Taslakta bu cümleden sonra yer alan hiçbir şey silinmemelidir.}}

\newpage

\hypertarget{references}{%
\section{Kaynakça}\label{references}}

\hypertarget{refs}{}
\begin{CSLReferences}{1}{0}
\leavevmode\vadjust pre{\hypertarget{ref-aydinonat:2007}{}}%
Aydınonat, N. E. (2007). İktisat Öğrencileri için Ödev Yazma Kılavuzu. \url{http://iktisat.cu.edu.tr/tr/Belgeler/Formlar/Bitirme\%20Projesi\%20Ödev\%20Hazırlama\%20Rehberi/N.\%20Emrah\%20AYDINONAT\%20(2006)\%20Ödev\%20Rehberi.pdf} adresinden erişildi.

\leavevmode\vadjust pre{\hypertarget{ref-ceylan2010yakinsama}{}}%
Ceylan, R. (2010). Yak{ı}nsama hipotezi: teorik tart{ı}{ş}malar. \emph{Sosyoekonomi}, \emph{11}(11).

\end{CSLReferences}

\end{document}
